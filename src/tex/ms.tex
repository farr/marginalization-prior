% Define document class
\documentclass[modern]{aastex631}
\usepackage{showyourwork}

\newcommand{\CCA}{Center for Computational Astrophysics, Flatiron Institute, New York NY 10010, USA}
\newcommand{\StonyBrook}{Department of Physics and Astronomy, Stony Brook University, Stony Brook NY 11794, USA}

% Begin!
\begin{document}

% Title
\title{A Useful Prior for Polarization Quadratures}

% Author list
\author{Max Isi}
\email{misi@flatironinstitute.org}
\affiliation{\CCA}

\author{Will M. Farr}
\email{wfarr@flatironinstitute.org}
\email{will.farr@stonybrook.edu}
\affiliation{\CCA}
\affiliation{\StonyBrook}


% Abstract with filler text
\begin{abstract}
    We say something about marginalization of quadratures with a Gaussian
    likelihood and a prior that is conditionally Gaussian but marginally flat
    (or any other shape desired).
\end{abstract}

% Main body with filler text
\section{Introduction}
\label{sec:intro}

Quadrature introduction, should probably introduce via \citet{Isi2022}.  

Because it makes analytical marginalization over quadratures easy
\citep{Hogg2020}, we often want to impose an isotropic Gaussian prior over
polarization quadratures with some scale parameter $A_0$:
\begin{equation}
    \label{eq:quadrature-prior}
    \mathbf{a} \sim N\left[ 0, a_0 \mathbf{I} \right].
\end{equation}
At fixed scale, however, this induces a prior on the amplitude $a \equiv \left|
\mathbf{a} \right|$ that has a power-law behavior as $a \to 0$:
\begin{equation}
    p\left( a \mid a_0 \right) = \frac{a^{n-1}}{2^{n/2-1} \Gamma\left(\frac{n}{2} \right) a_0^n} \exp\left[ -\frac{a^2}{2 a_0^2} \right],
\end{equation}
where $n$ is the number of quadrature components (i.e.\ $n = 4$ for
electromagnetic radiation or gravitational radiation in general relativity with
a sine and cosine quadrature for each of two polarizations in each Fourier
mode).

If we are trying to assess the presence of a mode this prior is inconvenient
because it places zero weight on $a = 0$ (i.e.\ the absence of the mode).  This
is purely a ``volume'' effect, arising from the fact that the multivariate
Gaussian prior in Eq.\ \eqref{eq:quadrature-prior} places finite weight at the
origin.  

If we promote $a_0$ to a parameter, with prior $p\left( a_0 \right)$ and use the
$n$ measurements---the components of $\mathbf{a}$---to constrain it, then by
tuning the prior on $a_0$ we can control the marginal prior on $a$.  To induce,
for example, a flat prior on $a$ as $a \to 0$, let the prior on $a_0$ be flat up
to some maximum (quadrature) scale $a_\mathrm{max}$.  Then 
\begin{equation}
    p\left( a, a_0 \right) = p\left( a \mid a_0 \right) p\left( a_0 \right) = \frac{a^{n-1}}{2^{n/2-1} \Gamma\left(\frac{n}{2} \right) a_0^n a_\mathrm{max}} \exp\left[ -\frac{a^2}{2 a_0^2} \right]
\end{equation}
induces a marginal prior on $a$ that is 
\begin{equation}
    p\left( a \right) \equiv \int \mathrm{d} a_0 \, p\left( a, a_0 \right) = \frac{\Gamma\left( \frac{n - 1}{2}, \frac{a^2}{2 a_\mathrm{max}^2} \right)}{a_\mathrm{max} \sqrt{2} \Gamma\left( \frac{n}{2} \right)} = \mathrm{const} + \mathcal{O}\left( \frac{a}{a_\mathrm{max}} \right)^{n-1}
\end{equation}

\begin{figure}
    \includegraphics{figures/prior_plot.pdf}
    \caption{The prior as a function of $n$.}
    \label{fig:prior}
    \script{prior_plot.py}
\end{figure}

\bibliography{bib}

\end{document}
